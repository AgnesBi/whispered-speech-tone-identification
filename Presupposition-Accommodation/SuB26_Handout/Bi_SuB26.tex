\documentclass[11pt,letterpaper]{scrartcl}
%\setkomafont{title}{\normalfont\bfseries} % Title in one line
\usepackage[margin=1in]{geometry}
\usepackage[T1]{fontenc}
\usepackage[parfill]{parskip}   % Activate to begin paragraphs with an empty line rather than an indent
\usepackage[none]{hyphenat} % prevent LaTeX from hyphenating at the end of a line

% ADJUST THE FORMAT OF SECTION HEADINGS
\usepackage[usenames,dvipsnames]{xcolor}
\definecolor{shadecolor}{gray}{0.85}
\usepackage{framed,fancybox}
\renewcommand{\thesection}{\Roman{section}}
\newcommand{\Section}[1]{\vspace{0.25cm} \begin{snugshade} \section{#1} \end{snugshade}}

% MISCELLANEOUS
\usepackage{comment}    % for multi-line comments

\usepackage{amsfonts,amssymb}
\usepackage[fleqn]{amsmath}
    \usepackage{mathtools}
    \usepackage{cases}
    \usepackage{bm} % bold in math mode
\usepackage{xcolor,soul}
\sethlcolor{gray!25}
\newcommand{\mathcolorbox}[2]{\colorbox{#1}{$\displaystyle #2$}}

\usepackage{graphicx}	% re-scale tree: \resizebox{}

\usepackage{tcolorbox}
\usepackage{pifont}
	\newcommand{\cmark}{\ding{51}} % check mark
	\newcommand{\xmark}{\ding{55}} % cross mark
\usepackage{enumitem}
    \newlist{todolist}{itemize}{2}
    \setlist[todolist]{label=$\square$}
	\newcommand{\done}{\rlap{$\square$}{\raisebox{2pt}{\large\hspace{1pt}\cmark}}% 
	\hspace{-2.5pt}}
    \newcommand{\wontfix}{\rlap{$\square$}{\large\hspace{1pt}\xmark}}
    
    \setlist[itemize]{leftmargin=*}
    \setlist[itemize,2]{label=$\triangleright$}
    \setlist[itemize,3]{label=--}

\usepackage[normalem]{ulem} % strikethrough text \sout{}

%%% Flow chart
\usepackage{tikz}
    \usetikzlibrary{decorations.pathreplacing,angles,quotes}
	\usetikzlibrary{shapes.geometric, arrows}
	\tikzstyle{startstop} = [rectangle, rounded corners, minimum width=3cm, minimum height=1cm,text centered, text width=5cm,draw=black, fill=gray!25]
	\tikzstyle{process} = [rectangle, rounded corners, minimum width=3cm, minimum height=1cm, text centered, text width=3.5cm,draw=black, fill=white]	
	\tikzstyle{arrow} = [thick,->,>=stealth]
\usepackage{adjustbox}

% REFERENCES
\usepackage{csquotes} 
\usepackage[style=apa,backend=biber,doi=false,isbn=false,url=false,eprint=false]{biblatex}
    \AtEveryBibitem{%
      \clearfield{note}%
    }
\addbibresource{references.bib}
\renewcommand*{\nameyeardelim}{\space}  % Author Year \cite{}

% SOME BASIC LINGUISTICS PACKAGES
% Phonology
\usepackage{tipa}
\usepackage{phonrule}
\usepackage[shadedcells]{ot-tableau}

% Syntax & Semantics
\usepackage{tikz-qtree}
\usepackage{tree-dvips}
\usepackage{semantic}
\usepackage{linguex}
\makeatletter
\newif\if@repeated\@repeatedfalse
\newcounter{savedExNo}
\renewcommand{\NormalEx}{\ifExWarning 
     \PackageWarning{linguex}{Check example numbering (screwed up?), 
     check number of empty lines at end of examples.  
     Detected}\fi\ExWarningtrue
     \if@repeated
        \Exformat[\ref{\tmp@ref}]
        \setcounter{ExNo}{\value{savedExNo}}
        \global\@repeatedfalse
     \else
     \if@noftnote\refstepcounter{ExNo}%
        \Exformat[\ExLBr\Exarabic{ExNo}\ExRBr]%
     \else
         \refstepcounter{FnExNo}\Exformat[\FnExLBr\Exroman{FnExNo}\FnExRBr]%
     \fi
     \fi}
\newcommand{\exr}[1]{%
\@repeatedtrue
\setcounter{savedExNo}{\value{ExNo}}
\def\tmp@ref{#1}
\ex.}
\makeatother        % cross referencing existing example

% USER-DEFINED COMMANDS
\makeatletter
\newcommand*{\rom}[1]{\expandafter\@slowromancap\romannumeral #1@}
\makeatother    % number -> roman numeral

\newcommand*{\TakeFourierOrnament}[1]{{%
\fontencoding{U}\fontfamily{futs}\selectfont\char#1}}
\newcommand*{\danger}{\TakeFourierOrnament{66}}     % \danger symbol

\newcommand{\alignright}{\hspace*{\fill}}
\newcommand{\infer}{$\rightsquigarrow$ }
\newcommand{\ninfer}{\( \cancel{\rightsquigarrow} \) }
\newcommand{\sub}[1]{\textsubscript{#1}}
\newcommand{\todo}[1]{{\color{magenta} #1}}
\newcommand{\lreqn}[2]{\noindent\makebox[.965\linewidth]{$\displaystyle#1$\hfill(#2)}\vspace{0ex}}

\newcommand{\interpfn}[1]{|[#1|]^{c, g, w}}

\title{Deriving evaluativity in \textit{even}-comparatives via presupposition accommodation}
%\subtitle{}
\author{Agnes Bi}
\date{GP Defense $|$ August 24, 2021}

\begin{document}
%\setlength{\Extopsep}{0pt}   % remove extra spacing before and after \ex. sentences
\setlength{\Exlabelsep}{0.6em}  % reduce spacing between \ex. label and \a. label
\setlength{\SubExleftmargin}{1.6em}  % reduce spacing between \a. label and text

% reduce spacing in align environments
\setlength{\abovedisplayskip}{0pt}
\setlength{\belowdisplayskip}{0pt}
\setlength{\abovedisplayshortskip}{0pt}
\setlength{\belowdisplayshortskip}{0pt}

\maketitle

\section{Puzzle}

\begin{itemize}
    \item This project investigates the inferences scalar focus particles such as \textit{even} trigger in comparative sentences

    \item There are two parts to this puzzle
        \begin{itemize}
            \item Standard sensitivity / Evaluativity: relative to a contextually determined standard
            
            \item Association with focus: inferences vary depending on the positioning of focal accent in the sentence
        \end{itemize}
\end{itemize}

%%%%%%%%%%%%%%%%%%
%%%%%%%%%%%%%%%%%%

\subsection{Evaluativity} 

\begin{itemize}
    \item Positive inference first observed in \cite{greenberg_even_2015}
    
    \ex. Alex is \textit{even} taller than Blake. \alignright [ \infer Both Alex and Blake are \textbf{tall}] \label{Greenberg10}
    
    \item Note that positive adjectives such as \textit{tall} by themselves do not produce evaluative inferences in the comparatives. We cannot infer how Alex's or Blake's height is compared to the standard from \Next alone:

    \ex. Alex is taller than Blake.
    
    \item The evaluative inference must have been triggered by the addition of \textit{even}
    
    \item Prior work has always assumed default pitch accent sentence-finally

\end{itemize}

%%%%%%%%%%%%%%%%%%
%%%%%%%%%%%%%%%%%%

\subsection{Association with focus}

\begin{itemize}
    \item The empirical landscape is actually more complicated
    
    \ex.\label{even-base} \a. Alex is even taller than [Blake]\textsubscript{F}.  % Blake is especially tall; Alex is taller yet. 
    \alignright [ \infer Both Alex and Blake are \textbf{tall}] \label{even-base-objF}
    \b. [Alex]\textsubscript{F} is even taller than Blake. % Alex is especially short, hence you don't expect him to be taller than anyone. But he is taller than Blake.
    \alignright [\infer Both Alex and Blake are \textbf{short}] \label{even-base-subF}
    
    \item We observe an unexpected scale-reversal in inferences when pitch accent is shifted to the target of the comparative
    
    \begin{itemize}
        \item \ovalbox{Terminology side-note} in the Degree literature, the subject of a comparative is referred to as the \textsc{target}, and the object of \textit{than}-clause in a phrasal comparative as the \textsc{standard} (not to confuse with the standard on a scale, which is conventionally understood as a small range around the median, see \cite{solt_notes_2011} for more details)
    \end{itemize}
    
    
\end{itemize}

\section{Existing semantic approach}

\begin{itemize}
    \item Positive inference \ref{Greenberg10} is taken as one of the main arguments to enrich the meaning of \textit{even} from its classic formulation (\cite{karttunen_conventional_1979}; \cite{rooth_theory_1992}; \cite{chierchia_logic_2013}; a.o.)

    \ex.\label{classicunlikelihood} \textbf{Classic (un)-likelihood analysis of \textit{even}} \\
        $|[\textit{even}|]^{g, c} = \lambda C. \lambda p : \forall q \in C \, [q \neq p \rightarrow p <_{\text{likely}} q]. p$ \\
        i.e., the prejacent \textit{p} is the least likely among its alternatives in the context C
    
    to a more involved one by explicitly encoding some version of a positive condition in its semantics
    
\end{itemize}

\subsection{\cite{greenberg_even_2015}: Gradability-based account}
\begin{itemize}
    \item A more radical revision by dispensing with the likelihood scale all together
    
    \item The presupposition of \textit{even} is instead imposed on a contextually determined scale \textsc{g}
    
    \item In addition to the comparison between the prejacent \textit{p} and its alternative \textit{q}, both \textit{p} and \textit{q} also need to have degrees that are at least as high as the standard on \textsc{g}
    
    \ex.\label{gradable} \textbf{Gradability-based analysis of \textit{even}} (Greenberg 2015, 2018)
    \begin{flalign*}
        |[\textit{even}|]^{g, c} = & \lambda C. \lambda p. \lambda w: \forall q \in C \big[q \neq p \rightarrow & \\ 
        & \forall w_1, w_2 [w_1Rw \land w_2Rw \land w_2 \in p \land w_1 \in [q \land \neg p]] \rightarrow & \\
        & [\textit{max}(\lambda d_2.G(d_2)(x)(w_2))) > \textit{max}(\lambda d_1.G(d_1)(x)(w_1)) & \text{Superlative Condition (a)} \\ 
        & \land \\
        %\boxed{
        & \mathcolorbox{gray!25}{\textit{max}(\lambda d_1.G(d_1)(x)(w_1)) \geqslant \textbf{Stand}_G }
        %}
        ] \big]. & \mathcolorbox{gray!25}{\text{Positive Condition (b)}} \\
        & p(w) = 1 & \text{Assertion}
    \end{flalign*}
    where \textit{x} is a non-focused entity within the prejacent \textit{p}, C is the set of alternatives, and \textsc{g} is a contextually supplied gradable property
    
    \item Applying this analysis to the puzzle
    
    \item For \ref{even-base-objF} where focus is on the object \textit{Blake}, following \cite{greenberg_revised_2018}, assume \textit{even} associates with the Degree Phrase \textit{-er than Blake}\footnote{I assume a phrasal comparison structure here for the ease of illustration. This is not crucial for Greenberg's analysis; phrasal comparison structure goes through in the same fashion.}. The set of alternatives C is then
    
    \ex. \begin{tabular}[t]{rl}
    $C_{\text{\ref{even-base-objF}}}$ & $=$ \{ Alex is as tall as Blake, Alex is taller than Blake \}\\
    & $=$ \{ $\max(\lambda d_1.\textsc{tall}(d_1)(\text{Alex})) \geq \max(\lambda d_2.\textsc{tall}(d_2)(\text{Blake}))$, \\
    & $\max(\lambda d_1.\textsc{tall}(d_1)(\text{Alex})) > \max(\lambda d_2.\textsc{tall}(d_2)(\text{Blake}))$ \}
    \end{tabular}
    
    \item Suppose the gradable predicate \textsc{g} measures degrees to which $x$ is tall
    
    \ex. \textbf{Presupposition of \ref{even-base-objF} per \cite{greenberg_revised_2018}}\\ 
    $\forall w_1, w_2 [w_1Rw \land w_2Rw \land w_2 \in [\max(\lambda d_1.\textsc{tall}(d_1)(\text{Alex})) > \max(\lambda d_2.\textsc{tall}(d_2)(\text{Blake}))]  \land w_1 \in [\max(\lambda d_1.\textsc{tall}(d_1)(\text{Alex})) = \max(\lambda d_2.\textsc{tall}(d_2)(\text{Blake}))]] \rightarrow$ \\
    $[\textit{max}(\lambda d_3.\textsc{tall}(d_3)(\text{Alex})(w_2))) > \textit{max}(\lambda d_1.\textsc{tall}(d_1)(\text{Alex})(w_1))$ \hspace*{\fill} Superlative Condition\\ 
    $\land \, \textit{max}(\lambda d_1.\textsc{tall}(d_1)(\text{Alex})(w_1)) \geqslant \text{Stand}_\textsc{tall} ] \big]$ \hspace*{\fill} Positive Condition
    
    \item When and only when Blake's height is known or fixed in the context, the superlative condition is guaranteed to be trivially satisfied  -- Alex's degree of tallness in all accessible worlds where she is taller than Blake (\textit{p} worlds) is higher than in all worlds where she is exactly Blake's height (\textit{q-and-not-p} worlds)
    
    \item The positive condition requires that Alex's degree of tallness in the worlds where she is the exactly same height as Blake to be at least as high as the standard for tallness $\rightarrow$ Blake is tall, and since Alex is taller than Blake, Alex is also tall
    
    \item The reasoning goes through roughly as expected, but the additional assumption of the height of Blake being fixed seems too restrictive
    
    \item Sentences such as \ref{even-base-subF} where the focus is on the subject, on the other hand, pose a bigger issue
    
    \item Assume \textit{even} associates with the subject \textit{Alex}. The set of alternatives $C$ is then

    \ex. $C_{\text{\ref{even-base-subF}}} =$ \{ \textsc{ALT(alex)} is taller than Blake, Alex is taller than Blake \}
    
    \item Considering the simplest case where $\{x | x \in \textsc{ALT(alex)}\}$ in this context is a singleton set containing only the individual \textit{Dominique}
    
    \item Suppose the gradable predicate \textsc{g} again measures degrees of tallness. The scalar presupposition of the sentence \ref{even-base-subF} is thus
    
    \ex. \textbf{Presupposition of \ref{even-base-subF} per \cite{greenberg_revised_2018} (attempt 1)} \\ 
    $\forall w_1, w_2 [w_1Rw \land w_2Rw \land w_2 \in [\max(\lambda d_1.\textsc{tall}(d_1)(\text{Alex})) > \max(\lambda d_2.\textsc{tall}(d_2)(\text{Blake}))]  
    \land 
    w_1 \in [\max(\lambda d_3.\textsc{tall}(d_3)(\text{Dominique})) >  \max(\lambda d_2.\textsc{tall}(d_2)(\text{Blake})) \, \land$ \\  \hspace*{\fill} $\max(\lambda d_1.\textsc{tall}(d_1)(\text{Alex})) \leq \max(\lambda d_2.\textsc{tall}(d_2)(\text{Blake}))]] \rightarrow
    $ \\
    $[\textit{max}(\lambda d_2.\textsc{tall}(d_2)(\text{Blake})(w_2))) > \textit{max}(\lambda d_1.\textsc{tall}(d_1)(\text{Blake})(w_1))$ \\ 
    $\land \, \textit{max}(\lambda d_1.\textsc{tall}(d_1)(\text{Blake})(w_1)) \geqslant \text{Stand}_{\textsc{tall}} ] \big]$
    
    \item The first conjunct requires that Blake's degree of tallness in all accessible worlds where she is shorter than Alex (\textit{p} worlds) is higher than in all worlds where she is shorter than Dominique but taller than Alex (\textit{q-and-not-p} worlds). Assuming Alex's height is known and fixed, it is always false, which wrongly predicates the sentence to be infelicitous
    
    \item Attempting to resolve this issue, let's try reversing the scale, accommodating a $\textsc{g}'$ measuring degrees of \textbf{shortness}
    
    \ex. \textbf{Presupposition of \ref{even-base-subF} per \cite{greenberg_revised_2018} (attempt 2)} \\ 
    $\forall w_1, w_2 [w_1Rw \land w_2Rw \land w_2 \in [\max(\lambda d_1.\textsc{short}(d_1)(\text{Alex})) > \max(\lambda d_2.\textsc{short}(d_2)(\text{Blake}))]  
    \land 
    w_1 \in [\max(\lambda d_3.\textsc{short}(d_3)(\text{Dominique})) >  \max(\lambda d_2.\textsc{short}(d_2)(\text{Blake})) \, \land$ \\  \hspace*{\fill} $\max(\lambda d_1.\textsc{short}(d_1)(\text{Alex})) \leq \max(\lambda d_2.\textsc{short}(d_2)(\text{Blake}))]] \rightarrow
    $ \\
    $[\textit{max}(\lambda d_2.\textsc{short}(d_2)(\text{Blake})(w_2))) > \textit{max}(\lambda d_1.\textsc{short}(d_1)(\text{Blake})(w_1))$ \\ 
    $\land \, \textit{max}(\lambda d_1.\textsc{short}(d_1)(\text{Blake})(w_1)) \geqslant \text{Stand}_{\textsc{short}} ] \big]$
    
    \item The first conjunct now requires that Blake's degree of \textbf{shortness} in all accessible worlds where she is shorter than Alex (\textit{p} worlds) is higher than in all worlds where she is shorter than Dominique but taller than Alex (\textit{q-and-not-p} worlds), which, assuming Alex's height is constant across all possible worlds, is trivially met
    
    \item The second conjunct requires that Blake's degree of shortness in the worlds where Alex is taller than her to surpass the standard for shortness, i.e., Blake is short
    
    \item It is important to note that the presupposition in \Last does not impose any condition on Alex's height relative to the standard
    
    \item We can only infer that Blake is short, but not that Alex is also short
    
    \item Unclear why this ad-hoc rescue strategy can be employed, and whether it is always available systematically
    
    \item In conclusion, the gradability-based account of \textit{even}
    
    \begin{itemize}
        \item is unsuccessful in capturing the complete inferences of comparative sentences with subject focus 
        
        \item imposes arbitrary restrictions on what needs to be known in the conversation background: the precise height of the focused element, Blake in \ref{even-base-objF} and Alex in \ref{even-base-subF}, needs to be known for the derivation to go through
        
        \item leaves open how the relevant \textsc{g} is determined, it remains a mystery why the scale is reversed for the minimal pair \ref{even-base}
    \end{itemize}
    
\end{itemize}

\subsection{\cite{daniels_even_2020}: Adding positive condition to the classic analysis}

\begin{itemize}
    \item Minimally revises the classic analysis \ref{classicunlikelihood}, hard-wiring a positive condition to the meaning of \textit{even}
    
    \ex.\label{DG_Unlikelihood} \textbf{Revised Comparative (un)Likelihood (Daniels and Greenberg 2020)} \\
    A proposition \textit{even p} is felicitous only if the following conditions hold:
    \a. \textbf{Superlative Condition}: \textit{p} is $<_{\text{likely}}$ \textit{q }, for all \textit{q} in C, where C is the set of contextually restricted focus alternatives;
    \b. \textbf{Positive Condition}: \textit{p} and \textit{q } are $>$ R\textsubscript{Std} for (un)likelihood, where R\textsubscript{Std} is the standard range, i.e., both \textit{p} and its alternatives \textit{q} are unlikely.
    
    \item Since the original wording left it unclear, let's assume the weakest, existential version of the Positive Condition, i.e., there exists a salient alternative \textit{q} such that both the prejacent \textit{p} and \textit{q} are unlikely
    
    \item To translate (un)likelihood to other scales, the following assumptions are made:

    \ex.\label{DG_assumptions} \textbf{\cite{daniels_even_2020}'s assumptions regarding scales of gradable predicates}
    \a. the standard range R\sub{Std} is an interval on the scale;
    \b. antonyms sit on the same scale, occupy the opposite sides outside of R\sub{Std};
    \b. having degrees within R\sub{Std} $>$\sub{likely} having degrees outside R\sub{Std};
    \b. having degrees within R\sub{Std} cannot be unlikely;
    \b.\label{averagevariations} variations within R\sub{Std} are allowed, i.e., it is possible to have a total ordering of the individuals fall within R\sub{Std}, but these differences are negligible, and thus not considered unlikely. 
    \z.

    \vspace{0.25cm}
    \begin{center}
    \resizebox{0.65\linewidth}{!}{
    \tikzset{every picture/.style={line width=0.75pt}} %set default line width to 0.75pt        
    \begin{tikzpicture}[x=0.75pt,y=0.75pt,yscale=-1,xscale=1]
    \linespread{1}% <--- locally defined vertical line spacing in nodes
    %---
    %Straight Lines [id:da6075768649538124] 
    \draw    (121.42,190) -- (536,190) ;
    \draw [shift={(538,190)}, rotate = 539.8299999999999] [color={rgb, 255:red, 0; green, 0; blue, 0 }  ][line width=0.75]    (10.93,-3.29) .. controls (6.95,-1.4) and (3.31,-0.3) .. (0,0) .. controls (3.31,0.3) and (6.95,1.4) .. (10.93,3.29)   ;
    %Straight Lines [id:da45568459744739576] 
    \draw  [dash pattern={on 0.84pt off 2.51pt}]  (268,156.25) -- (268,208.25) ;
    %Straight Lines [id:da6210541845865379] 
    \draw  [dash pattern={on 0.84pt off 2.51pt}]  (372,156.25) -- (372,208.25) ;
    %Straight Lines [id:da9793154130726737] 
    \draw [color={rgb, 255:red, 65; green, 117; blue, 5 }  ,draw opacity=1 ]   (378,140) -- (526.42,140) ;
    \draw [shift={(528.42,140)}, rotate = 180.1] [color={rgb, 255:red, 65; green, 117; blue, 5 }  ,draw opacity=1 ][line width=0.75]    (10.93,-3.29) .. controls (6.95,-1.4) and (3.31,-0.3) .. (0,0) .. controls (3.31,0.3) and (6.95,1.4) .. (10.93,3.29)   ;
    %Straight Lines [id:da5752384790946132] 
    \draw [color={rgb, 255:red, 65; green, 117; blue, 5 }  ,draw opacity=1 ]   (248,140) -- (124.42,140) ;
    \draw [shift={(122.42,140)}, rotate = 360.34000000000003] [color={rgb, 255:red, 65; green, 117; blue, 5 }  ,draw opacity=1 ][line width=0.75]    (10.93,-3.29) .. controls (6.95,-1.4) and (3.31,-0.3) .. (0,0) .. controls (3.31,0.3) and (6.95,1.4) .. (10.93,3.29)   ;
    %Curve Lines [id:da3838445384619129] 
    \draw [decorate,decoration={brace,amplitude=5pt}, color={rgb, 255:red, 74; green, 144; blue, 226 } ] (268,140) -- (372,140);
    
    \draw (514,204) node [anchor=north west][inner sep=0.75pt]   [align=left] {Height};
    \draw (315,190) node [circle,fill,inner sep=1.5pt,label=below:\text{\footnotesize Median}] {};
    \draw (300,155) node [anchor=north west][inner sep=0.75pt]   [align=left] {R\sub{std}};
    \draw (268,80) node [anchor=north west][inner sep=0.75pt]  [font=\footnotesize, color={rgb, 255:red, 74; green, 144; blue, 226 }] [text width=8em, align=left] {falling in this range is NOT considered unlikely};
    \draw (153,155) node [anchor=north west][inner sep=0.75pt]   [align=left] {\textit{pos} \textsc{short}};
    \draw (432,155) node [anchor=north west][inner sep=0.75pt]   [align=left] {\textit{pos} \textsc{tall}};
    
    \draw (123,120) node [anchor=north west][inner sep=0.75pt]  [font=\small, color={rgb, 255:red, 65; green, 117; blue, 5 }] [align=left] {increasingly unlikely};
    \draw (393,120) node [anchor=north west][inner sep=0.75pt]  [font=\small, color={rgb, 255:red, 65; green, 117; blue, 5 }] [align=left] {increasingly unlikely};
    
    \end{tikzpicture}
    }
    \end{center}

    \vspace{0.25cm}
    
    \item For \ref{even-base-objF}, the alternative set $C = \{\text{Alex is taller than Blake, Alex is taller than Cassie ...}\}$. The authors assume that the Superlative Condition inherited from the classic comparative (un)likelihood analysis guarantees the ordering Alex $>_{\text{tall}}$ Blake $>_{\text{tall}}$ Cassie \danger
    
    \item The additional Positive Condition requires both $p = $ \textit{Alex is taller than Blake} and $q =$ \textit{Alex is taller than Cassie} be unlikely. Since an individual can be tall, short, or neither, we have $3 \times 3 = 9$ possible cases regarding Alex's and Blake's height status
    
    \vspace{0.5cm}
    \begin{tabular}{cc|l}
        Alex & Blake & \\ \hline
        \textit{tall} & \textit{tall} & ? \\
        \textit{tall} & \textit{average} & ? \\
        \sout{\textit{tall}} & \sout{\textit{short}} & since it's not unlikely to be taller than a short individual, \textit{p} is not unlikely \\ \hline
        \sout{{\color{gray}\textit{average}}} & \sout{{\color{gray}\textit{tall}}} & {\color{gray} contradicts Alex $>_{\text{tall}}$ Blake}\\
        \sout{\textit{average}} & \sout{\textit{average}} & following assumption \Last[e], \textit{p} is not unlikely\\
        \sout{\textit{average}} & \sout{\textit{short}} & since it's not unlikely to be taller than a short individual, \textit{p} is not unlikely \\ \hline
        \sout{{\color{gray}\textit{short}}} & \sout{{\color{gray}\textit{tall}}} & {\color{gray} contradicts Alex $>_{\text{tall}}$ Blake} \\
        \sout{{\color{gray}\textit{short}}} & \sout{{\color{gray}\textit{average}}} & {\color{gray} contradicts Alex $>_{\text{tall}}$ Blake} \\
        \sout{\textit{short}} & \sout{\textit{short}} & since it's not unlikely to be taller than a short individual, \textit{p} is not unlikely
    \end{tabular} 
    \vspace{0.5cm}
    
    \item As the authors point out, the two cases with ? need further examination: 
    
    \begin{itemize}
        \item If Alex is tall and Blake is neither tall or short, one could argue \textit{p} is not unlikely if we assume that, within a default distribution, the existence of individuals having degrees outside the standard range is expected. This is another, albeit sensible, assumption that needs independent support
        
        \item If both Alex and Blake are tall, since variations are expected among the tall individuals, \textit{p} is not necessarily unlikely. Their proposal would then wrongly predicate a presupposition failure
    \end{itemize}
    
    \item Daniels and Greenberg's proposal is a reasonable and straightforward attempt to address the puzzle of standard sensitivity in \textit{even}-comparatives. However, it suffers from two major issues
        
        \begin{itemize}
            \item The analysis takes for granted the likelihood ranking of the propositions \textit{p} and \textit{q} derives where the individuals sit relative to one another on the relevant scale
                \begin{itemize}
                    \item Nothing in our current system guarantees this mapping; in theory, there can be many reasons as to why \textit{p} is less likely than \textit{q}
                    
                    \item Consider \ref{even-base-objF} in a more specific context
                    
                    \ex. \label{sibling} \textit{Context: The speaker is talking about the three kids in the Smith family. So far, it has been established that Alex is taller than her twin Aaron and that Blake is their older sibling.} \\
                    Alex is even taller than [Blake]\sub{F}. \alignright cf. \ref{even-base-objF}
                    %This is only felicitous when the the height ordering between Aaron and Blake is unknown, or Blake is taller than Aaron
                    
                    \item $p =$ \textit{Alex is taller than Blake} is less likely than $q =$ \textit{Alex is taller than Aaron} simply because that it's more likely for a kid to be taller than someone of their own age than to be taller than someone older
                    
                    \item Importantly, it doesn't have to the case that Blake is strictly taller than Aaron in actuality
                    
                    \item A separate, well-defined machinery is needed to explain why and how likelihoods of propositions are translated to degrees on a scale
                    
                \end{itemize}
            
            \item The derivation relies on the assumptions in \ref{DG_Unlikelihood}, which are satisfied by relative adjectives but not absolute adjectives (\cite{kennedy_vagueness_2007}, \cite{toledo_absolute_2011}, \cite{lassiter_context_2013})
            
            \begin{itemize}
                \item As a result, the same reasoning cannot apply to comparable sentences containing absolute adjectives
                
                \item But empirically speaking, absolute adjectives behave the same way as relative adjectives with respect to their inference patterns
                
                \ex.\label{safe} \textit{Maximum standard absolute adjective}
                \a. District A is even safer than [District B]\sub{F}. \alignright [\infer Both districts are \textbf{safe}]
                \b. [District A]\sub{F} is even safer than District B. \alignright [\infer Both districts are \textbf{dangerous}]
                
                \ex.\label{dirty} \textit{Minimum standard absolute adjective}
                \a. The study is even dirtier than [the kitchen]\sub{F}. \alignright [\infer Both rooms are \textbf{dirty}]
                \b. [The study]\sub{F} is even dirtier than the kitchen. \alignright [\infer Both rooms are \textbf{clean}]
                
                \item This would have to be explained either by proposing a separate mechanism or by revising the assumptions for \cite{daniels_even_2020}
                
            \end{itemize}
            
                
        \end{itemize}
    
    \end{itemize}
    
    \vspace{0.25cm}
    \begin{tcolorbox}[colback=yellow!5!white,colframe=yellow!75!black,title=Interim summary]
    
    The existing semantic analyses 
    
    \vspace{2pt}
    
    \begin{itemize}[itemsep=2pt]
        \item fail to capture the complete inference patterns observed in \textit{even}-comparatives;
        
        \item are unsatisfactory conceptually considering the stipulative assumptions;
        
        \item cannot easily explain the systematic scale reversal effect triggered by the change in focus placement;
        
        \item contingent on the mechanism assumed for setting the standard of a gradable predicate, they seem to predict that, since the positive condition is hardwired, evaluative inferences always arise in comparative sentences with \textit{even}, which is not borne out empirically. 
        
        \begin{itemize}
            \item In the sibling scenario \ref{sibling}, the Smith kids can be tall, short or average height; we don't necessarily infer anything about their absolute height status from the sentence
        \end{itemize}
        
    \end{itemize}

    \tcblower
    
    It is also interesting to note that similar evaluative inferences are seen in other scalar particles such as \textit{at least} and \textit{still}, and it would be ideal to have an uniform analysis, instead of enriching their semantics individually
    
    \ex. \a. The study is at least cleaner than [the kitchen]\sub{F}. \alignright{[\infer Both rooms are \textbf{dirty}]}
    \b. Simon is taller still than Paige. \alignright{[\infer Both Simon and Paige are \textbf{tall}]}

    \end{tcolorbox}

\section{My proposal: a pragmatic approach}

\begin{itemize}

    \item \underline{Main idea}: supplementing the classic (un)-likelihood analysis of \textit{even} with two general pragmatic principles can derive these inferences systematically
    
    \ex. \textbf{Presupposition Accommodation Condition (PAC)}\\
    In cases where a presupposition is not entailed by the common ground, listeners can accommodate it only if its best explanation is already entailed or can be accommodated at the same time. % Q: best explanation of what?? Best explanation of the assertion
    
    \ex. \textbf{Alternative-sampling Hypothesis (ASH)} \\ When the alternative set is not explicitly specified, assume that individuals included in the computation of alternatives form a representative sample of the contextually determined relevant population with respect to \textsc{g}.
    
    \item \underline{Assumptions}:
        \begin{itemize}
            \item Alternative semantics approach (\cite{rooth_association_1985}) to focus-sensitivity
                
                \begin{itemize}
                    \item the general function of focus is to evoke alternatives, which are sets of propositions
                \end{itemize}
 
            \item \textit{even} always takes scope over matrix TP and associates with focus
        \end{itemize}
    
    \item For \ref{even-base}, according to the classic (un)likelihood analysis of \textit{even} \ref{classicunlikelihood}, both sentences assert that \textit{Alex is taller than Blake}, but different alternative sets are invoked:

    \ex. Alternatives of \ref{even-base-objF} =  \{Alex is taller than $\pi$ | $\pi \in \textsc{ALT(Blake)}$\} \\
    Alternatives of \ref{even-base-subF} =  \{$\pi$ is taller than Blake | $\pi \in \textsc{ALT(Alex)}$\}
    
    \item The presupposition of \textit{even} mandates that the proposition \textit{Alex is taller than Blake} is the least likely among its alternatives
    
    \item Either this presupposition is already entailed by the common ground, in which case no accommodation is needed and evaluative inference do not necessarily arise, as in \ref{sibling}; or, the addressee learns that this likelihood ranking is entailed by the speaker's intended common ground
    
    \item In the later case, somewhat unexpectedly, the addressee simultaneously commences an abductive reasoning trying to find the most likely justification for this ranking
    
    \begin{itemize}
        \item Discourse participants are not satisfied with simply accommodating the likelihood presupposition on its own; they need to understand why the speaker would presuppose such differences in likelihood exist in the first place
    \end{itemize}
    
    \item I claim that in the absence of further context, absolute height status is the best and most salient explanation
    
    \ex. \textbf{Abduction} (\cite{douven_abduction_2017}) \\ Given evidence $E$ and candidate explanations $H_1, ..., H_n$ of $E$, infer the truth of \textit{that} $H_i$ which best explains $E$.
    
    \begin{itemize}
        \item For \ref{even-base-objF}, the evidence \textit{E} is the fact the speaker's intended common ground entails that \textit{Alex is taller than Blake} is the least likely proposition among its alternatives \{Alex is taller than $\pi$\}
        
        \item There can be many reasons as to why that would be the case. To name a few candidate explanations
        
        \begin{itemize}[nosep]
            \item $H_1 =$  Blake is the oldest in the group of teenagers
            
            \item $H_2 =$ Blake is a mythical giant
            
            \item $H_3 = $ Blake is the only modern human compared to a group of mid-19th century people
            
            \item etc.
        \end{itemize}
        
        \item Among these explanations, $H_i$ Blake is the tallest within the contextually determined group is the one that presumes the least from context, which, I hypothesize, makes it the ``best" explanation of \textit{E}
            
            \begin{itemize}
                \item The link between degrees and likelihoods is a natural one, considering the \textit{transitive} and \textit{antisymmetric} properties of ordering relations
                
                \ex. For a gradable predicate \textsc{g} with responding scale $>_\textsc{g}$%, and $x, y, z \in \mathfrak{F}(>_\textsc{g})$,
                \a. Given $ x >_\textsc{g} y$. If $z >_\textsc{g} x$, then $z >_\textsc{g} y$;
                \b. Given $ y >_\textsc{g} x$. If $x >_\textsc{g} z$, then $y >_\textsc{g} z$.
                
                \item For example, if Ben is taller than Carl, and Ali is taller than Ben, than Ali is taller than Carl
                
                \item Given that if $\phi$ entails $\psi$, $\phi$ is at most as likely as $\psi$
                
                \ex. For a gradable predicate \textsc{g} with responding scale $>_\textsc{g}$, \\ \vspace{0.25cm}
                $
                \begin{cases}
                    \lreqn{x >_\textsc{g} y \to \mathcal{L}\big( z >_\textsc{g} x \big) \leq \mathcal{L}\big( z >_\textsc{g} y\big)}{a} \\
                    \lreqn{y >_\textsc{g} x \to \mathcal{L}\big( x >_\textsc{g} z \big) \leq \mathcal{L}\big( y >_\textsc{g} z\big)}{b}
                \end{cases}
                $
                
            \end{itemize}
        
    \end{itemize}
    
    \item Via abduction, listeners infers $H_i$ to be true, i.e., Blake is the tallest within the set of the individuals considered in the computation of alternatives
    
    \begin{itemize}
        \item We will refer to this set as the set of individuals under consideration, and it forms the comparison class for the interpretation of the gradable predicate
    \end{itemize}
    
    \item Still not quite there yet! Blake being the tallest within the comparison class does not necessarily mean Blake is tall. Perhaps everyone else in the comparison class is exceptionally short, in which case we still do not know Blake's height status relative to the standard
    
    \item To bridge the gap, we need the additional assumption that when an \textit{even}-comparative is uttered more or less out of blue, i.e., when the individuals under consideration are not specified in the discourse, the comparison class is representative of the contextually determined population (formalized as Alternative-sampling Hypothesis)
    
    \item At its core, all ASH claims is for interlocutors to assume normality unless there is evidence indicating otherwise
    
    \item Now we have all the necessary pieces to give a complete derivation of the positive inference in \ref{even-base-objF} when uttered in a context that does not already support the presupposition of \textit{even}
    
    \begin{figure}[h]
      \vspace{0.25cm}
      \begin{center}
        \resizebox{\textwidth}{!}{
        \begin{tikzpicture}[node distance=6cm]
        	\node (start) [startstop] {$p =$ \textit{Alex is taller than Blake} is less likely than its alternatives  \{Alex is taller than $\pi$\}};
        	\node (step1) [process, right of=start] {Blake is taller than everyone else in the comparison class};
        	\node (step2) [process, right of=step1] {Blake is tall};
        	\node (step3) [process, right of=step2] {Alex is tall};
        	\draw [arrow] (start) -- node[anchor=south] {PAC} (step1);
        	\draw [arrow] (step1) -- node[anchor=south] {ASH}(step2);
        	\draw [arrow] (step2) -- node[anchor=south] {$p = 1$}(step3);
        \end{tikzpicture}
         }
         \end{center}
         \caption{Pragmatic reasoning process for \ref{even-base-objF}}
    \end{figure}
    
    \item The reversed, negative, inference of \ref{even-base-subF} falls out straightforwardly from the same reasoning process
    
    \begin{figure}[h]
      \vspace{0.25cm}
      \begin{center}
        \resizebox{\textwidth}{!}{
        \begin{tikzpicture}[node distance=6cm]
        	\node (start) [startstop] {$p =$ \textit{Alex is taller than Blake} is less likely than its alternatives \{$\pi$ is taller than Blake\} };
        	\node (step1) [process, right of=start] {Alex is shorter than everyone else in the comparison class};
        	\node (step2) [process, right of=step1] {Alex is short};
        	\node (step3) [process, right of=step2] {Blake is short};
        	\draw [arrow] (start) -- node[anchor=south] {PAC} (step1);
        	\draw [arrow] (step1) -- node[anchor=south] {ASH}(step2);
        	\draw [arrow] (step2) -- node[anchor=south] {$p = 1$}(step3);
        \end{tikzpicture}
        }
      \end{center}
      \caption{Pragmatic reasoning process for \ref{even-base-subF}}
    \end{figure}

\end{itemize}

\vspace{0.25cm}
    \begin{tcolorbox}[colback=yellow!5!white,colframe=yellow!75!black,title=Summary]
    
    A pragmatic approach
    
    \vspace{2pt}
    
    \begin{todolist}[itemsep=2pt]
        \item[\done] captures the complete inference patterns observed in \textit{even}-comparatives
        \item[\done] requires few specific, stipulative assumptions
        \item[\done] easily explains the systematic scale reversal effect
        \item[\done] accounts for optionality of the evaluative inferences
    \end{todolist}
    
    \tcblower
    
    Future research
    \begin{todolist}
        \item extending to other scalar particles, both in English and crosslinguistically
    \end{todolist}

    \end{tcolorbox}


%\section{Red herring: relative vs. absolute adjective distinction}

%\begin{itemize}
%    \item 
%\end{itemize}

%\section{Where to go from here}

%\begin{itemize}
    %\item 
%\end{itemize}







%%%%%%%%%%%%%%%%%%%%%%%%%%%%%%%%%%%%%%%%%%%%%%%%%%%%%%%%%%%%%%%%%%%%%%%%
\newpage
\printbibliography

\end{document}
